\documentclass[a4paper, 12pt]{article}
\usepackage[top=2cm, bottom=2cm, left=2.5cm, right=2.5cm]{geometry}
\usepackage[T1]{fontenc}
\usepackage[utf8]{inputenc}
\usepackage[brazil]{babel}
\usepackage{graphicx}
\usepackage{pdfpages}
\usepackage{enumerate}

\graphicspath{{Imagens/}}
\begin{document}

%///////////////////////////BEGIN CAPA////////////////////////////
\begin{titlepage}
    \begin{center}

        {\normalsize \bfseries UNIVERSIDADE FEDERAL DE MINAS GERAIS}\\[2cm]
        {\normalsize Labolatório de Análise Circuitos I (ELE028)}\\[2cm]
        {\normalsize Gabriel Luiz de Almeida e Felipe Meireles Leonel}\\[8cm]
        {\large \bfseries Movimentos Harmônicos Simples}\\[10cm]
        {\normalsize Belo Horizonte, Minas Gerais\\2023}\\[4cm]

    \end{center}
\end{titlepage}
%///////////////////////////END CAPA//////////////////////////////

%///////////////////////////BEGIN OBJETIVOS////////////////////////////
\section{Objetivos}

\paragraph{} Calcular o comprimento de onda da luz emitida por um laser e estabelecer a
relação entre o índice de refração do ar e a pressão atmosférica.
%////////////////////////////END OBJETIVOS/////////////////////////////

%///////////////////////////BEGIN MATERIAIS////////////////////////////
\section{Materiais}
\begin{itemize}
    \item laser de Ne-He
    \item interferômetro de Michelson
    \item câmara transparente e bomba de vácuo
\end{itemize}
%////////////////////////////END MATERIAIS/////////////////////////////

%///////////////////////////BEGIN PROCEDIMENTO////////////////////////////
\section{Procedimento}

\subsection{Parte 1}
\begin{enumerate}[(a)]
    \item Primeiramente o laser foi ligado, e os equipamentos calibradoos de tal forma
          que se observa-se anéis iluminados na tela do interferômetro.
    \item Com o materias prontos e montados, girou-se o micrômetro do interferômetro a
          fim de se contar 100 máximos de luz no centro dos anéis da tela.
    \item Observando o deslocamento necessário do espelho a partir do micrômetro
          necessário de para a passagem de 100 máximos na tela, determinou-se o
          comprimento de onda do laser utilizado.
\end{enumerate}

\subsection{Parte 2}
\begin{enumerate}[(a)]
    \item Para a segunda parte do experimetro, aproveitou-se a montagem dos equipamentos
          acrescentando-se uma câmara posicionada logo a frente do laser.
    \item Retirou-se então o ar da câmara, até que se tivesse uma pressão negatica de 800
          mili Bar, em relação a pressão ambiente.
    \item Observando o máximos de luz no centro dos anéis na tela, liberou-se a entrada
          de pressão para dentro a câmara. Contou-se a quantidade de máximos de luz que
          passavam de 100 em 100 mili bar, anotando-os a fim de se obter um gráfico
          posteriormente.
    \item Usando um software de plotagem de gráfico, realizou-se uma regressão linear com
          os dados obtidos. Com os resultados da regressão, foram realizados calculos
          para a determinação do indice de refração do ar.
\end{enumerate}
%////////////////////////////END PROCEDIMENTO/////////////////////////////

%///////////////////////////BEGIN RESULTADOS////////////////////////////
\section{Resultados}
\paragraph{} Do experiento foi obtido o seguinte gráfico da força sentida pelo sensor ao
longo do tempo.

\begin{figure}[!htb]
    \begin{center}
        \includegraphics[scale=.60]{dados}\label{grafico 1}
        \caption{ Gráfico das forças medidas do movimento hamonico simples ao longo do tempo}
    \end{center}
\end{figure}
\newpage
\pagebreak

\paragraph{} A partir do ajuste de seno realizado no gráfico 1, obteve-se o gráfico 2, que
contem a função que melhor descreve o movimento do sistema massa mola.

\begin{figure}[!htb]
    \begin{center}
        \includegraphics[scale=.60]{Ajuste}\label{grafico 2}
        \caption{ Gráfico do ajuste de seno realizado}
    \end{center}
\end{figure}
% \newpage
% \pagebreak

\paragraph{} Das informações obtidas pelo ajuste de seno do gráfico do ajuste, foram
realizados os seguintes calculos para a obtenção do valor da constante elástica
da mola.

\newpage
\pagebreak
\includepdfmerge[pages=-]{pdf/calculos}
%////////////////////////////END RESULTADOS/////////////////////////////

%//////////////////////BEGIN ANÁLISE E DISCURSÕES///////////////////////
\section{Análise e discursões}

\paragraph{} Com base nos resultados obtidos, podemos concluir que a mola em questão é
composta principalmente de ferro, o que era uma expectativa razoável, dadas as
características do experimento.

\paragraph{} Além disso, vale destacar que o método utilizado para determinar a constante
elástica da mola tem aplicações mais amplas, incluindo a capacidade de estimar
a massa de um objeto em um contexto de movimento harmônico simples. Essa
técnica poderia, por exemplo, ser empregada para calcular a massa de um veículo
através das oscilações de sua suspensão. Essa abordagem é particularmente
valiosa em competições de automobilismo, onde a análise dinâmica de um veículo
em diversos aspectos é crucial para o seu desempenho e otimização.
%//////////////////////END ANÁLISE E DISCURSÕES/////////////////////////

%///////////////////////////BEGIN CONCLUSÃO////////////////////////////
\section{Conclusão}
\paragraph{} Conclui-se então que o experimento foi realizado com sucesso, uma vez que a
constante elástica obtida se aproximou bastante com o valor esperado.
%///////////////////////////END CONCLUSÃO////////////////////////////

%///////////////////////////BEGIN ANEXOS////////////////////////////
% \section{Conclusão}
% \paragraph{} Conclui-se então que o experimento foi realizado com sucesso, uma vez que a
% constante elástica obtida se aproximou bastante com o valor esperado.
%///////////////////////////END ANEXOS////////////////////////////
\end{document}